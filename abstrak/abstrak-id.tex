\begin{center}
  \large\textbf{ABSTRAK}
\end{center}

\addcontentsline{toc}{chapter}{ABSTRAK}

\vspace{2ex}

\begingroup
  % Menghilangkan padding
  \setlength{\tabcolsep}{0pt}

  \noindent
  \begin{tabularx}{\textwidth}{l >{\centering}m{2em} X}
    % Ubah kalimat berikut dengan nama mahasiswa
    Nama Mahasiswa    &:& Dimas Nazli Bahaduri \\

    % Ubah kalimat berikut dengan judul tugas akhir
    Judul Tugas Akhir &:&	\emph{Ether Wallet} Menggunakan \emph{Ethereum} Berbasis \emph{Proof of Stake} \\

    % Ubah kalimat-kalimat berikut dengan nama-nama dosen pembimbing
    Pembimbing        &:& 1. Mochamad Hariadi, ST., M.Sc., Ph.D. \\
  &:& 2. Dr. Supeno Mardi Susiki Nugroho, ST., MT.
  \end{tabularx}
\endgroup

% Ubah paragraf berikut dengan abstrak dari tugas akhir
Pada penelitian ini kami mengajukan penelitian yang memanfaatkan perkembangan \emph{Ethereum} sebagai salah satu solusi perbankan yaitu dompet digital. Bagian perbankan mengalami perkembangan dengan mengadopsi dompet digital yang dinilai cukup bagus bagi konsumen karena meminimalisir transaksi tak tercatat. Sebagai solusi terpusatnya digunakanlah \emph{Blockchain Ethereum} sebagai sistem transaksi utama. \emph{Ethereum} ini mengadopsi sistem informasi terdistribusi yang mana mengharuskan setiap audit informasi dibagikan ke semua node yang terhubung. Alhasil dengan sistem informasi seperti ini, segala transaksi dan akses informasi bisa terlihat ke semua node yang membuat sulit untuk diubah. Nantinya dengan mengadopsi Ethereum ke dalam dompet digital baru diharapkan bisa mengeliminasi masalah yang mungkin ditemui dari dompet konvensional dan memberikan opsi baru bagi pengguna konsumen di Indonesia. Penelitian ini akan menjalankan proses transaksi yang diawali dengan permintaan untuk melakukan transaksi. Ketika proses transaksi dimulai akan dilakukan verifikasi data transaksi dari penerima dan mengirim. Setelah transaksi terverifikasi, selanjutkan proses validasi dari node – node tergabung dengan jaringan \emph{Ethereum}. Validasi ini akan menggunakan konsensus/kesepakatan \emph{Proof of Stake} yang mana menganalisa apakah benar pengirim mempunyai sejumlah nominal yang akan dikirimkan. Kemudian jika validasi sudah mencapai ambang batas,sejumlah nominal tercantum dalam transaksi akan dikirimkan dari pengirim ke penerima. Maka proses transaksi tersebut selesai. Diharapkan dari penelitian ini bisa membuka potensi pemanfaatan \emph{Ethereum} dan bisa dimanfaatkan oleh masyarakat luas.

% Ubah kata-kata berikut dengan kata kunci dari tugas akhir
Kata Kunci: Dompet digital, \emph{Ethereum}, \emph{Proof of Stake}.
