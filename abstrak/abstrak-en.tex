\begin{center}
  \large\textbf{ABSTRACT}
\end{center}

\addcontentsline{toc}{chapter}{ABSTRACT}

\vspace{2ex}

\begingroup
  % Menghilangkan padding
  \setlength{\tabcolsep}{0pt}

  \noindent
  \begin{tabularx}{\textwidth}{l >{\centering}m{3em} X}
    % Ubah kalimat berikut dengan nama mahasiswa
    \emph{Name}     &:& Dimas Nazli Bahaduri \\

    % Ubah kalimat berikut dengan judul tugas akhir dalam Bahasa Inggris
    \emph{Title}    &:& \emph{Ether Wallet using Ethereum based on Proof of Stake} \\

    % Ubah kalimat-kalimat berikut dengan nama-nama dosen pembimbing
    \emph{Advisors} &:& 1. Mochamad Hariadi, ST., M.Sc., Ph.D. \\
  &:& 2. Dr. Supeno Mardi Susiki Nugroho, ST., MT.
  \end{tabularx}
\endgroup

% Ubah paragraf berikut dengan abstrak dari tugas akhir dalam Bahasa Inggris
\emph{In this research, we proposed the usage and harnessing development of Ethereum as one of internet banking solution. As a growing sector in Indonesia, internet banking eliminate a lot of problems that might occurs during using the conventional banking such as unrecorded transaction, limited availability, and many more. While using Ethereum as platform could eliminate these problems. Furthermore Ethereum using choices of consensus such as Proof of Stake which validates the information transaction from the nodes. Using selected consensus could enforce the security of information and the selected nodes are given small amount of ETH as reward (usually stated in gas used). Later, by adopting Ethereum into a new digital wallet, it is hoped that it will eliminate problems that may be encountered from conventional wallets and provide new options for consumer users in Indonesia. This research will run a transaction process that begins with a request to make a transaction. When the transaction process starts, it will verify the transaction data from the recipient and send. After the transaction is verified, continue the validation process from the nodes joined to the Ethereum network. This validation will use a consensus Proof of Stake which analyzes whether it is true that the sender has a nominal amount to be sent. Then if the validation has reached the threshold, the nominal amount listed in the transaction will be sent from the sender to the recipient. Then the transaction process is complete. It is hoped that this research can unlock the potential for the use of Ethereum and can be used by the wider community. }

% Ubah kata-kata berikut dengan kata kunci dari tugas akhir dalam Bahasa Inggris
\emph{Keywords}: \emph{Digital Wallet} , \emph{Ethereum} , \emph{Proof of Stake}.
